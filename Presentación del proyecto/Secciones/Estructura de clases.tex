\section{Estructura de clases}

\subsection{Process Documents}
\begin{frame}{Process Documents}
    Esta clase se encarga de procesar los documentos de la base de datos antes de comenzar la búsqueda.
    Se llama al método en la línea anterior a app.Run () para realizar esta operación una sola vez antes de que
    arranque el programa y esté listo para ejecutarse.

\pause

    \begin{center}
        \includegraphics[width=8cm, height=0.75cm]{fig 1.1.png}\\
    \end{center}
\end{frame}  

\subsection{Documents}

\begin{frame}
    \frametitle{Documents}
    Esta clase esta dirigida al trabajo con los documentos, se “normalizan”. Es decir, en caso de que las
busquedas se realicen en espaol, se eliminan las tildes para evitar errores ortogr´aficos. Ademas, independientemente del idioma, se eliminan los signos de puntuacion y cualquier otro sımbolo ajeno al alfabeto 
y los numeros

\pause

\begin{center}
	\includegraphics[width=11cm, height=3cm]{fig 1.3.png}\\
\end{center}
\end{frame}

\begin{frame}{Process Documents}
    Se encuentra ademas esta otra funcion para seleccionar el “snippet” (pedazo breve de un texto) que se 
    imprimira en el momento de la busqueda por cada resultado.

\pause

\begin{center}
	\includegraphics[width=10.7cm, height=5cm]{fig 1.4.png}\\
\end{center}
\end{frame}

\subsection{Matrix}
\begin{frame}{Matrix}
    El valor de “relevancia” de una palabra esta dado por el calculo de su TF (Term Frequency) por su IDF
(Inverse Document Frequency), para ello se ha utilizado la formula:

\begin{center}
	$\frac{nd}{Cd} \cdot \log (\frac{T}{N})$ \\
\end{center}

Donde:
\

\begin{itemize}
    \item nd es la cantidad de ocurrencias de una palabra en un documento,
\pause 
    \item Cd es la cantidad total de palabras en el documento,
\pause
    \item T es la cantidad total de documentos,
\pause
    \item N es la cantidad de documentos en los que aparece la palabra.
\end{itemize}
\end{frame}

\subsection{Moogle}
\begin{frame}{Moogle}
    Esta es la clase principal del programa, donde comienza el proceso de busqueda. Comienza en el momento
en que se recibe la query.

\pause

\begin{center}
	\includegraphics[width=7cm, height=2.7cm]{fig 1.10 (2).png}\\
\end{center}
\end{frame}

\begin{frame}{Moogle}
    Se iteran las palabras sin repetir de la query y se calcula su TF-IDF en los
documentos en los que aparece, pero para conseguir el score por documento se necesita la suma de estos
valores.

\

El score queda de la siguiente manera: si la palabra esta contenida en el documento que se esta analizando se encontrara el valor de TF-IDF
correspondiente a esa palabra en ese documento, de forma contraria el valor sera 0.

\end{frame}